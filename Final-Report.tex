\PassOptionsToPackage{unicode=true}{hyperref} % options for packages loaded elsewhere
\PassOptionsToPackage{hyphens}{url}
%
\documentclass[]{article}
\usepackage{lmodern}
\usepackage{amssymb,amsmath}
\usepackage{ifxetex,ifluatex}
\usepackage{fixltx2e} % provides \textsubscript
\ifnum 0\ifxetex 1\fi\ifluatex 1\fi=0 % if pdftex
  \usepackage[T1]{fontenc}
  \usepackage[utf8]{inputenc}
  \usepackage{textcomp} % provides euro and other symbols
\else % if luatex or xelatex
  \usepackage{unicode-math}
  \defaultfontfeatures{Ligatures=TeX,Scale=MatchLowercase}
\fi
% use upquote if available, for straight quotes in verbatim environments
\IfFileExists{upquote.sty}{\usepackage{upquote}}{}
% use microtype if available
\IfFileExists{microtype.sty}{%
\usepackage[]{microtype}
\UseMicrotypeSet[protrusion]{basicmath} % disable protrusion for tt fonts
}{}
\IfFileExists{parskip.sty}{%
\usepackage{parskip}
}{% else
\setlength{\parindent}{0pt}
\setlength{\parskip}{6pt plus 2pt minus 1pt}
}
\usepackage{hyperref}
\hypersetup{
            pdftitle={Report},
            pdfauthor={Margaret Gacheru, Melanie Mayer, Kee-Young Shin, Adina Zhang},
            pdfborder={0 0 0},
            breaklinks=true}
\urlstyle{same}  % don't use monospace font for urls
\usepackage[margin=1in]{geometry}
\usepackage{graphicx,grffile}
\makeatletter
\def\maxwidth{\ifdim\Gin@nat@width>\linewidth\linewidth\else\Gin@nat@width\fi}
\def\maxheight{\ifdim\Gin@nat@height>\textheight\textheight\else\Gin@nat@height\fi}
\makeatother
% Scale images if necessary, so that they will not overflow the page
% margins by default, and it is still possible to overwrite the defaults
% using explicit options in \includegraphics[width, height, ...]{}
\setkeys{Gin}{width=\maxwidth,height=\maxheight,keepaspectratio}
\setlength{\emergencystretch}{3em}  % prevent overfull lines
\providecommand{\tightlist}{%
  \setlength{\itemsep}{0pt}\setlength{\parskip}{0pt}}
\setcounter{secnumdepth}{0}
% Redefines (sub)paragraphs to behave more like sections
\ifx\paragraph\undefined\else
\let\oldparagraph\paragraph
\renewcommand{\paragraph}[1]{\oldparagraph{#1}\mbox{}}
\fi
\ifx\subparagraph\undefined\else
\let\oldsubparagraph\subparagraph
\renewcommand{\subparagraph}[1]{\oldsubparagraph{#1}\mbox{}}
\fi

% set default figure placement to htbp
\makeatletter
\def\fps@figure{htbp}
\makeatother


\title{Report}
\author{Margaret Gacheru, Melanie Mayer, Kee-Young Shin, Adina Zhang}
\date{May 10, 2020}

\begin{document}
\maketitle

\hypertarget{introduction}{%
\section{Introduction}\label{introduction}}

Hurricanes are type of storms with high speed winds that form over
tropical or subtropical waters. Typically, hurricanes bring about strong
winds, storm surges, and heavy rainfall that can lead to flooding,
tornadoes, etc. The Saffir-Simpson Hurricane scale is used to rate
hurricanes by their wind speed -- the higher the category, the greater
the possibility of landfall damage. It is worthwhile to understand
aspects of hurricanes and make accurate predictions in order to provide
appropriate information and resources to the public, make informed
decisions, and potentially save lives.

For this project, we have information about 356 hurricanes in the North
Atlantic area since 1989, including the storm's location and maximum
windspeed for every 6 hours. Additionally, the dataset includes
information about season and month which hurricane occured, nature of
the hurricane, and timing of the storm. We are interested in building a
bayesian model in order to predict hurricane trajectories, particularly
the wind speed at a specific time point. We will utilize Markov Chain
Monte-Carlo to estimate the model parameters and their 95\% credible
intervals through the posterior distributions. Furthermore, we are
interested in evaluating our model's ability to accurately predict
windspeed and track the hurricanes.

\hypertarget{methods}{%
\section{Methods}\label{methods}}

\hypertarget{bayesian-and-mcmc-basics}{%
\subsection{Bayesian and MCMC Basics}\label{bayesian-and-mcmc-basics}}

Unlike frequentist approaches, Bayesian analysis treats parameters as
random variables and utilizes prior beliefs about the parameters. We
start out with a prior distribution of the parameter of interest and
then update it with observed data to obtain the posterior distribution.
With the posterior distribution, we can obtain parameters estimates such
as posterior mean. Bayes Theorem allows us to observe the relationship
between prior distribution (\(\pi(\theta)\)), likelihood
(\(f(x|\theta)\)), and posterior distribution (\(\pi(\theta|x)\))

\[ \pi(\theta|x) = \cfrac{f(x|\theta) \pi(\theta)}{m(x)} = 
\cfrac{f(x|\theta) \pi(\theta)}{\int f(x|\theta) \pi(\theta) d\theta}\]

Obtaining the posterior distribution is often a difficult task -- either
there is no closed form or it is computationally intensive to directly
sample from \(\pi(\theta|x)\). Therefore, MCMC is used to create a
Markov chain of the parameter of interest such that their distribution
converge to the posterior.

\hypertarget{hurricane-bayesian-model}{%
\subsection{Hurricane Bayesian Model}\label{hurricane-bayesian-model}}

The Bayesian model used to predict hurricane trajectories can be
expressed as follows: \(Y_i(t+6)=\mu_i(t)+\rho_iY_i(t)+\epsilon_i(t)\)
where \(\rho_j\) is the autoregressive correlation and the error
\(\epsilon_i\) follows the normal distribution with mean 0 and variance
\(\sigma^2\) independent across t. \(\mu_i(t)\) represents the function
mean that can be written as follows:
\[\mu_i(t) = \beta_0+\beta_1x_{i,1}(t)+\beta_2x_{i,2}(t)+\beta_3x_{i,3}(t)+\sum^3_{k=1}\beta_{3+k}\triangle_{i,k}(t-6)\]
where \(x_{i,1}(t)\), \(x_{i,2}(t)\), and \(x_{i,3}(t)\) are the day of
the year, calendar year, and type of hurricane at time t, respectively.
and \[\Delta_{i,k}(t-6) = Y_{i,k}(t) -Y_{i,k}(t-6),k=1,2,3\] are the
change of latitude, longitude, and wind speed from time \(t-6\) to
\(t\). The following prior distributions were assumed:
\(\pi(\boldsymbol{\beta})\) is jointly normal with mean 0 and variance
\(diag(1,p)\), \(\pi(\rho_j)\) follows a truncated normal
\(N_{[0,1]}(0.5, 1/5)\), and \(\pi(\sigma^{-2})\) follows an
inverse-gamma \((0.001, 0.001)\).

\hypertarget{metropolis-hastings-algorithm}{%
\subsection{Metropolis-Hastings
Algorithm}\label{metropolis-hastings-algorithm}}

To estimate the parameters, a component-wise Metropolis-Hastings (MH)
algorithm was suggested, with a uniform proposal for \(\beta\) and
\(\rho\), and a inverse-gamma proposal for \(\sigma^2\). The MH
algorithm constructs a Markov chain by accepting candidate points with
probability
\[\alpha(y|x^{(t)})=min(1,\frac{\pi(y)q(x^{(t)}|y)}{\pi(x^{(t)})q(y|x^{(t)})})\]
The random-walk Metropolis is a special class where the proposed
transition q is symmetric. As a result, the acceptance probability is
only related to \(\pi(y)\), i.e.~prior distributions. The response wind
speed was assumed to follow a normal distribution and the log-likelihood
could be expressed as
\[l(\textbf{\beta}, \rho)=-\frac{n}{2}log(2\pi)-nlog(\sigma)-\frac{1}{2\sigma^2}\sum^n_{i=1}(y-X^T\beta-\rho Y)^2\]
The partial posterior distribution can be formed by the product of the
log-likelihood and the priors. For the MH algorithm, a burn in was
incorporated to ensure analysis of a stationary distribution.

\hypertarget{results}{%
\section{Results}\label{results}}

\hypertarget{discussion}{%
\section{Discussion}\label{discussion}}

\end{document}
